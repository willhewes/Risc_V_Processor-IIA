\documentclass[a4paper,10pt]{article}
% Use ctrl + alt + V to view live pdf

% Packages
\usepackage[utf8]{inputenc} % For encoding
\usepackage[T1]{fontenc} % Better handling of accented characters and hyphenation
\usepackage{microtype} % Improves spacing and justification
\usepackage{amsmath, amssymb} % For equations and symbols
\usepackage{graphicx} % For including graphics/images
\usepackage{caption} % For customizing figure and table captions
\usepackage{subcaption} % For subfigures and subcaptions
\usepackage{float} % For fixing figure and table positions
\usepackage{booktabs} % For professional-looking tables
\usepackage{siunitx} % For consistent typesetting of units and numbers
\usepackage[margin=2cm]{geometry} % Adjusts page margins
\usepackage{fancyhdr} % For custom headers and footers
\usepackage{lmodern} % For a professional-looking font (main body font)
\usepackage{titlesec} % For title customization
\usepackage{array} % For custom table formatting
\usepackage[colorlinks=true, linkcolor=black, urlcolor=black]{hyperref} % Colored links without boxes
\usepackage{cleveref} % For improved cross-referencing    
\usepackage{multirow}
\usepackage{enumitem}
\usepackage{listings}
\usepackage{xcolor}
\usepackage{textcomp}
\usepackage{tabularx}
\usepackage{changepage}
\usepackage{tikz}
\usetikzlibrary{shapes.geometric, arrows}

\lstdefinestyle{vhdl-style}{
    language=VHDL,
    basicstyle=\ttfamily\footnotesize,
    keywordstyle=\bfseries\color{blue},
    commentstyle=\itshape\color{gray},
    stringstyle=\color{red},
    numbers=left,
    numberstyle=\tiny\color{gray},
    stepnumber=1,
    breaklines=true,
    showstringspaces=false,
    frame=single
}
\lstset{style=vhdl-style}
\lstset{captionpos=b}
\lstset{basicstyle=\ttfamily\scriptsize} 
\renewcommand{\lstlistingname}{Program}

% Custom settings
\pagestyle{fancy}
\fancyhf{}
\fancyhead[L]{\textit{GB3 - Risc-V Processor}} % Header left
\fancyhead[R]{\textit{Will Hewes - wh365}} % Header right 
\fancyfoot[C]{\thepage} % Footer center
\setlength{\headheight}{15pt} % Header height
\setlength{\parindent}{0em} % Indentation for paragraphs
\setlength{\parskip}{0.5em} % Add spacing between paragraphs
\setlength{\abovedisplayskip}{1em}
\setlength{\belowdisplayskip}{1em}
\setlength{\abovedisplayshortskip}{1em}
\setlength{\belowdisplayshortskip}{1em}
% \setlist{topsep=0em, partopsep=0em, itemsep=0em, parsep=0em}

\graphicspath{{C:/Users/willi/My Drive/Engineering/Obsidian Vault/Uni/Lab Reports/3B2/FTR/Images}}

\renewcommand{\arraystretch}{1.2}

% Title formatting
\renewcommand{\maketitle}{
    \begin{center}
        \LARGE \textbf{ENGINEERING TRIPOS PART IIA} \\ 
        \vspace{0.5em}
        \Large \textbf{GB3 - Risc-V Processor} \\ 
        \vspace{0.5em}
        \textbf{First Interim Report} \\
        \large Group 4 - Resource Usage \\
        \vspace{1em}
        \large Will Hewes - wh365 \\ 
        Pembroke College \\ 
        \vspace{0.5em}
    \end{center}
}

\begin{document}
\maketitle
\hrule
\tableofcontents
\newpage

\section{Introduction}
\label{sec:Introduction}
% This section should give a quick overview of the team’s task as a whole

This project involves the collaborative design and implementation 
of a RISC-V processor on an iCE40 FPGA platform. 
The overarching aim is to optimise 
the processor design with respect to its 
performance, power dissipation, and resource usage, 
while balancing trade-offs between these three aspects. 
Each team member is responsible for focusing on one of the three aspects.

This interim report focuses on the initial characterisation of the baseline design. 
We are evaluating three provided example programs - 
\texttt{hardwareblink}, \texttt{softwareblink}, and \texttt{bubblesort} - 
to establish benchmarks for the logic usage, power dissipation and timing. 
These programs exercise the processor and surrounding hardware to varying degrees, 
and provide a foundation against which future design improvements can be compared.


\section{Project Specification}
\label{sec:Project_Specification}
%This section should provide an overview of your specific
%task as part of the team.

Within the team, my specific task focuses on 
optimising the resource usage of the FPGA implementation. 
The goal of this optimisation is to reduce the number of 
Look-Up Tables (LUTs), flip-flops, and wires used in the design. 
This allows more headroom on the device for future expansions 
and increases the scalability of the design.

LUTs, flip-flops, and routing wires are 
the fundamental building blocks of digital logic in an FPGA. 
A LUT is a configurable logic element that can implement 
arbitrary combinational logic functions 
by storing precomputed outputs for each possible input combination. 
Flip-flops are used to store single bits of state and 
are essential for implementing sequential logic, 
such as registers and control signals. 
Routing wires connect LUTs and flip-flops together, 
allowing signals to propagate through the design. 
The quantity and efficiency of these elements directly impact 
how much logic the FPGA can implement and how complex the system can be.

For this interim report, the goal is not to 
change the programs or optimise the device yet.
This report only characterises the resource usage of the three programs.


\section{Summary of Preliminary Design Work}
\label{sec:Summary_of_Preliminary_Design_Work}


\subsection{Hardwareblink}
\label{sec:Hardwareblink}

The first of these programs is the \texttt{hardwareblink} program,
which causes \textit{LED D14} to blink at a rate of 1Hz. 
This is acheived by ...

\subsection{Softwareblink}
\label{sec:Softwareblink}

\subsection{Bubblesort}
\label{sec:Bubblesort}

\section{Conclusion and Future Work}
\label{sec:Conclusion_and_Future_Work}

\appendix
\section{Appendix}
%Use this section to include diagrams, Verilog or C code, etc

\end{document}