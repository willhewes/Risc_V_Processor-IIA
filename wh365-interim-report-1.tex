\documentclass[a4paper,10pt]{article}
% Use ctrl + alt + V to view live pdf

% Packages
\usepackage[utf8]{inputenc} % For encoding
\usepackage[T1]{fontenc} % Better handling of accented characters and hyphenation
\usepackage{microtype} % Improves spacing and justification
\usepackage{amsmath, amssymb} % For equations and symbols
\usepackage{graphicx} % For including graphics/images
\usepackage{caption} % For customizing figure and table captions
\usepackage{subcaption} % For subfigures and subcaptions
\usepackage{float} % For fixing figure and table positions
\usepackage{booktabs} % For professional-looking tables
\usepackage{siunitx} % For consistent typesetting of units and numbers
\usepackage[margin=2cm]{geometry} % Adjusts page margins
\usepackage{fancyhdr} % For custom headers and footers
\usepackage{lmodern} % For a professional-looking font (main body font)
\usepackage{titlesec} % For title customization
\usepackage{array} % For custom table formatting
\usepackage[colorlinks=true, linkcolor=black, urlcolor=black]{hyperref} % Colored links without boxes
\usepackage{cleveref} % For improved cross-referencing    
\usepackage{multirow}
\usepackage{enumitem}
\usepackage{listings}
\usepackage{xcolor}
\usepackage{textcomp}
\usepackage{tabularx}
\usepackage{changepage}
\usepackage{tikz}
\usetikzlibrary{shapes.geometric, arrows}
\newcolumntype{Y}{>{\centering\arraybackslash}X}


\lstdefinestyle{vhdl-style}{
    language=VHDL,
    basicstyle=\ttfamily\footnotesize,
    keywordstyle=\bfseries\color{blue},
    commentstyle=\itshape\color{gray},
    stringstyle=\color{red},
    numbers=left,
    numberstyle=\tiny\color{gray},
    stepnumber=1,
    breaklines=true,
    showstringspaces=false,
    frame=single
}
\lstset{style=vhdl-style}
\lstset{captionpos=b}
\lstset{basicstyle=\ttfamily\scriptsize} 
\renewcommand{\lstlistingname}{Program}

% Custom settings
\pagestyle{fancy}
\fancyhf{}
\fancyhead[L]{\textit{GB3 - Risc-V Processor}} % Header left
\fancyhead[R]{\textit{Will Hewes - wh365}} % Header right 
\fancyfoot[C]{\thepage} % Footer center
\setlength{\headheight}{15pt} % Header height
\setlength{\parindent}{0em} % Indentation for paragraphs
\setlength{\parskip}{0.2em} % Add spacing between paragraphs
\setlength{\abovedisplayskip}{0.5em}
\setlength{\belowdisplayskip}{0.5em}
\setlength{\abovedisplayshortskip}{0.5em}
\setlength{\belowdisplayshortskip}{0.5em}
% \setlist{topsep=0em, partopsep=0em, itemsep=0em, parsep=0em}

\graphicspath{{Images/}}

% \renewcommand{\arraystretch}{1.2}

% Title formatting
\renewcommand{\maketitle}{
    \begin{center}
        \LARGE \textbf{ENGINEERING TRIPOS PART IIA} \\ 
        \vspace{0.5em}
        \Large \textbf{GB3 - Risc-V Processor} \\ 
        \vspace{0.5em}
        \textbf{First Interim Report} \\
        \large Group 4 - Resource Usage \\
        \vspace{1em}
        \large Will Hewes - wh365 \\ 
        Pembroke College \\ 
        \vspace{0.5em}
    \end{center}
}

\begin{document}
\maketitle
\hrule
\tableofcontents
\newpage

\section{Introduction}
\label{sec:Introduction}
% This section should give a quick overview of the team’s task as a whole

This project involves the collaborative design and implementation 
of a RISC-V processor on an iCE40 FPGA platform. 
The overarching aim is to optimise 
the processor design with respect to its 
performance, power dissipation, and resource usage, 
while balancing trade-offs between these three aspects. 
Each team member is responsible for focusing on one of the three aspects.
This report will focus on the resource usage of the design.

This interim report characterises the baseline design. 
We are evaluating three provided example programs - 
\texttt{hardwareblink}, \texttt{softwareblink}, and \texttt{bubblesort} - 
to establish benchmarks for the logic usage, power dissipation and timing. 
These programs exercise the processor and surrounding hardware to varying degrees, 
and provide a foundation against which future design improvements can be compared.


\section{Project Specification}
\label{sec:Project_Specification}
%This section should provide an overview of your specific
%task as part of the team.

Within the team, my specific task focuses on 
optimising the resource usage of the FPGA implementation. 
The goal of this optimisation is to reduce the number of 
Look-Up Tables (LUTs), flip-flops, and logic elements used in the design. 
This allows more headroom on the device for future expansions 
and increases its scalability.

For this interim report, the goal is not to 
change the programs or optimise the device yet.
This report only characterises the resource usage of the three programs.


\section{Summary of Preliminary Design Work}
\label{sec:Summary_of_Preliminary_Design_Work}

To assess the resource usage of each of the three programs, 
the designs were synthesised using \texttt{yosys} 
and place-and-route was performed using \texttt{nextpnr}. 

\subsection{Hardwareblink}
\label{sec:Hardwareblink}

The first program evaluated was \texttt{hardwareblink}.
This example does not use the RISC-V processor core and 
instead relies purely on combinational and 
sequential logic defined in \texttt{Verilog}.

This program implements a simple hardware-only design that 
toggles \textit{LED D14} at 1\,Hz using a counter. 
In order to toggle the LED at 1\,Hz, 
\texttt{hardwareblink} makes use of the 48\,MHz \textit{HFOSC},
and divides it by 24\,000\,000 in a counter.
This output toggles a register, 
the value of which is mapped to pin \textit{D3} (driving \textit{LED D14}),
giving rise to a 1\,Hz blinking light.

The summary of the resulting logic utilisation is shown
in Table \ref{tab:hardware_yosys_report} below.

\begin{table}[H]
    \centering
    \begin{tabular}{|l|c|}
        \hline
        \textbf{Primitive} & \textbf{Count} \\
        \hline
        \texttt{SB\_LUT4} logic cells (combinational) & 65 \\
        \texttt{SB\_DFF} flip-flops & 0 \\
        \texttt{SB\_DFFE} flip-flops with clock enable & 1 \\
        \texttt{SB\_DFFSR} flip-flops with set/reset & 32 \\
        \texttt{SB\_DFFSS} flip-flops with set/set & 0 \\
        \texttt{SB\_CARRY} carry-chain elements & 61 \\
        \texttt{SB\_RAM40\_4K} block RAMs & 0 \\
        \texttt{SB\_HFOSC} internal oscillator & 1 \\
        \hline
    \end{tabular}
    \caption{Primitive cell usage reported by 
    \texttt{yosys} for the \texttt{hardwareblink} design}
    \label{tab:hardware_yosys_report}
\end{table}



Corresponding to 160 cells in total.

Subsequently, \texttt{nextpnr} was used to place and route these logic cells,
with the results shown in Table \ref{tab:hardware_pnr_report}.

\begin{table}[H]
    \centering
    \begin{tabularx}{0.6\textwidth}{X c c}
        \toprule
        Resource type & Used & \% of total \\ \midrule
        Logic cells (\texttt{ICESTORM\_LC}) & 103 / 5\,280 & 1.9\,\% \\
        Block RAMs (\texttt{ICESTORM\_RAM}) & 0 / 30 & 0\,\% \\
        IO buffers (\texttt{SB\_IO}) & 1 / 96 & 1.0\,\% \\
        Global buffers (\texttt{SB\_GB}) & 2 / 8  & 25\,\% \\
        HF oscillators (\texttt{SB\_HFOSC}) & 1 / 1  & 100\,\% \\ 
    \bottomrule
    \end{tabularx}
    \caption{Place-and-route logic utilisation reported by 
    \texttt{nextpnr} for \texttt{hardwareblink}}
    \label{tab:hardware_pnr_report}
\end{table}

This report shows that only 103 logic cells were used on the FPGA fabric, 
corresponding to approximately 2\% of the available 5280 logic cells 
on the device.


No block RAMs, memories, or processor logic were involved in this design, 
making it a minimal baseline for comparison. 
The low utilisation confirms the simplicity of the circuit 
and provides a reference point for understanding 
how resource usage increases with more complex processor-driven programs.


\subsection{Softwareblink}
\label{sec:Softwareblink}

The second program evaluated was \texttt{softwareblink}. 
In contrast to \texttt{hardwareblink}, 
this design makes use of the full RISC-V processor core 
to execute a compiled C program that toggles \textit{LED D14} in software. 
The program operates by writing to a memory-mapped I/O register at regular intervals, 
with timing determined by a simple delay loop. 
This introduces a significant increase in logic complexity, 
as instruction decoding, arithmetic, control logic, and memory access logic 
are all active.

\iffalse
To synthesise this design, \texttt{softwareblink.c} was compiled 
using the RISC-V GCC toolchain into \texttt{program.hex} and \texttt{data.hex}, 
which were then loaded into the processor design 
before synthesis using \texttt{yosys} and place-and-route using \texttt{nextpnr}.
\fi

Table \ref{tab:software_yosys_report} shows the breakdown of primitive cell usage 
as reported by \texttt{yosys} after synthesis.

\begin{table}[H]
    \centering
    \begin{tabular}{|l|c|}
        \hline
        \textbf{Primitive} & \textbf{Count} \\
        \hline
        \texttt{SB\_LUT4} logic cells (combinational) & 1\,965 \\
        \texttt{SB\_DFF} flip-flops & 415 \\
        \texttt{SB\_DFFE} flip-flops with clock enable & 154 \\
        \texttt{SB\_DFFSR} flip-flops with set/reset & 100 \\
        \texttt{SB\_DFFSS} flip-flops with set/set & 4 \\
        \texttt{SB\_CARRY} carry-chain elements & 216 \\
        \texttt{SB\_RAM40\_4K} block RAMs & 20 \\
        \texttt{SB\_HFOSC} internal oscillator & 1 \\
        \hline
    \end{tabular}
    \caption{Primitive cell usage reported by 
    \texttt{yosys} for the \texttt{softwareblink} processor design}
    \label{tab:software_yosys_report}
\end{table}

Table \ref{tab:software_pnr_report} shows the corresponding logic usage 
on the FPGA fabric as reported by \texttt{nextpnr}. 
This design consumes 50\% of the device's logic cell resources 
and two-thirds of its available block RAMs, 
representing a substantial increase in resource usage 
relative to \texttt{hardwareblink}.

\begin{table}[H]
    \centering
    \begin{tabularx}{0.6\textwidth}{X c c}
        \toprule
        Resource type & Used & \% of total \\ \midrule
        Logic cells (\texttt{ICESTORM\_LC}) & 2662 / 5280 & 50\,\% \\
        Block RAMs (\texttt{ICESTORM\_RAM})& 20 / 30 & 66\,\% \\
        IO buffers (\texttt{SB\_IO}) & 8 / 96 & 8\,\% \\
        Global buffers (\texttt{SB\_GB}) & 5 / 8 & 62\,\% \\
        HF oscillators (\texttt{ICESTORM\_HFOSC}) & 1 / 1 & 100\,\% \\
        \bottomrule
    \end{tabularx}
    \caption{Place-and-route logic utilisation reported by 
    \texttt{nextpnr} for \texttt{softwareblink}}
    \label{tab:software_pnr_report}
\end{table}

This evaluation establishes the baseline resource cost 
of running programs on the processor. 
The increased logic and memory usage, relative to \texttt{hardwareblink}, 
reflect the added complexity of 
instruction decoding, pipeline control, and memory access. 
These results provide a benchmark for future optimisations 
aimed at reducing the resource footprint of the processor.

\subsection{Bubblesort}
\label{sec:Bubblesort}

The third program evaluated was \texttt{bubblesort}, 
which tests the processor under a significantly 
heavier computational load compared to \texttt{softwareblink}. 
The program implements a nested loop to sort a small array in memory, 
exercising the arithmetic, comparison, branching, 
and memory access capabilities of the RISC-V core. 
This makes it a more demanding benchmark for 
evaluating resource utilisation.

\iffalse
As with the previous example, 
\texttt{bubblesort.c} was compiled into 
\texttt{program.hex} and \texttt{data.hex}, 
which were loaded into the processor before synthesis. 
Table \ref{tab:bubblesort_yosys_report} shows the resulting breakdown of 
primitive cells as reported by \texttt{yosys}.
\fi

\begin{table}[H]
    \centering
    \begin{tabular}{|l|c|}
        \hline
        \textbf{Primitive} & \textbf{Count} \\
        \hline
        \texttt{SB\_LUT4} logic cells (combinational) & 2\,344 \\
        \texttt{SB\_DFF} flip-flops & 415 \\
        \texttt{SB\_DFFE} flip-flops with clock enable & 154 \\
        \texttt{SB\_DFFSR} flip-flops with set/reset & 100 \\
        \texttt{SB\_DFFSS} flip-flops with set/set & 4 \\
        %\texttt{SB\_DFFESR} flip-flops with enable + set/reset & 2 \\
        %\texttt{SB\_DFFN} negative edge flip-flop & 1 \\
        \texttt{SB\_CARRY} carry-chain elements & 216 \\
        \texttt{SB\_RAM40\_4K} block RAMs & 20 \\
        \texttt{SB\_HFOSC} internal oscillator & 1 \\
        \hline
    \end{tabular}
    \caption{Primitive cell usage reported by 
    \texttt{yosys} for the \texttt{bubblesort} processor design}
    \label{tab:bubblesort_yosys_report}
\end{table}

The corresponding place-and-route results reported by 
\texttt{nextpnr} are shown in Table \ref{tab:bubblesort_pnr_report}.


\begin{table}[H]
    \centering
    \begin{tabularx}{0.6\textwidth}{X c c}
        \toprule
        Resource type & Used & \% of total \\ \midrule
        Logic cells (\texttt{ICESTORM\_LC}) & 3073 / 5280 & 58\,\% \\
        Block RAMs (\texttt{ICESTORM\_RAM}) & 20 / 30 & 66\,\% \\
        IO buffers (\texttt{SB\_IO}) & 8 / 96 & 8\,\% \\
        Global buffers (\texttt{SB\_GB}) & 5 / 8 & 62\,\% \\
        HF oscillators (\texttt{ICESTORM\_HFOSC}) & 1 / 1 & 100\,\% \\
        \bottomrule
    \end{tabularx}
    \caption{Place-and-route logic utilisation reported by 
    \texttt{nextpnr} for \texttt{bubblesort}}
    \label{tab:bubblesort_pnr_report}
\end{table}

This design required a larger number of LUTs than 
\texttt{softwareblink}, increasing from 1\,965 to 2\,344, 
as the processor executes a more instruction-heavy and 
branching-intensive routine. 
However, the number of flip-flops, RAM blocks, 
and carry-chain elements remained constant. 
This suggests that the core hardware logic is unchanged, 
and the increase in combinational logic reflects 
the execution of a more complex program 
rather than architectural modifications.

\section{Conclusion and Future Work}
\label{sec:Conclusion_and_Future_Work}

\appendix
\section{Appendix}
%Use this section to include diagrams, Verilog or C code, etc

\end{document}