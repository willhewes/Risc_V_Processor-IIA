\documentclass[a4paper,10pt]{article}
% Use ctrl + alt + V to view live pdf

% Packages
\usepackage[utf8]{inputenc} % For encoding
\usepackage[T1]{fontenc} % Better handling of accented characters and hyphenation
\usepackage{microtype} % Improves spacing and justification
\usepackage{amsmath, amssymb} % For equations and symbols
\usepackage{graphicx} % For including graphics/images
\usepackage{caption} % For customizing figure and table captions
\usepackage{subcaption} % For subfigures and subcaptions
\usepackage{float} % For fixing figure and table positions
\usepackage{booktabs} % For professional-looking tables
\usepackage{siunitx} % For consistent typesetting of units and numbers
\usepackage[margin=2cm]{geometry} % Adjusts page margins
\usepackage{fancyhdr} % For custom headers and footers
\usepackage{lmodern} % For a professional-looking font (main body font)
\usepackage{titlesec} % For title customization
\usepackage{array} % For custom table formatting
\usepackage[colorlinks=true, linkcolor=black, urlcolor=black]{hyperref} % Colored links without boxes
\usepackage{cleveref} % For improved cross-referencing    
\usepackage{multirow}
\usepackage{enumitem}
\usepackage{listings}
\usepackage{xcolor}
\usepackage{textcomp}
\usepackage{tabularx}
\usepackage{changepage}
\usepackage{tikz}
\usepackage{pdfpages}
\usetikzlibrary{shapes.geometric, arrows}
\newcolumntype{Y}{>{\centering\arraybackslash}X}
% Reduce spacing before and after \section
\titlespacing{\section}{0pt}{1.0em}{0.5em}

% Reduce spacing before and after \subsection
\titlespacing{\subsection}{0pt}{0.8em}{0.3em}


\lstdefinestyle{vhdl-style}{
    language=VHDL,
    basicstyle=\ttfamily\footnotesize,
    keywordstyle=\bfseries\color{blue},
    commentstyle=\itshape\color{gray},
    stringstyle=\color{red},
    numbers=left,
    numberstyle=\tiny\color{gray},
    stepnumber=1,
    breaklines=true,
    showstringspaces=false,
    frame=single
}
\lstset{style=vhdl-style}
\lstset{captionpos=b}
\lstset{basicstyle=\ttfamily\scriptsize} 
\renewcommand{\lstlistingname}{Program}

% Custom settings
\pagestyle{fancy}
\fancyhf{}
\fancyhead[L]{\textit{GB3 - Risc-V Processor}} % Header left
\fancyhead[R]{\textit{Will Hewes - wh365}} % Header right 
\fancyfoot[C]{\thepage} % Footer center
\setlength{\headheight}{15pt} % Header height
\setlength{\parindent}{0em} % Indentation for paragraphs
\setlength{\parskip}{0.2em} % Add spacing between paragraphs
\setlength{\abovedisplayskip}{0.5em}
\setlength{\belowdisplayskip}{0.5em}
\setlength{\abovedisplayshortskip}{0.5em}
\setlength{\belowdisplayshortskip}{0.5em}
\setlist{topsep=0em, partopsep=0em, itemsep=0em, parsep=0em}

\graphicspath{{Images/}}

% \renewcommand{\arraystretch}{1.2}

% Title formatting
\renewcommand{\maketitle}{
    \begin{center}
        \LARGE \textbf{ENGINEERING TRIPOS PART IIA} \\ 
        \vspace{0.5em}
        \Large \textbf{GB3 - Risc-V Processor} \\ 
        \vspace{0.5em}
        \textbf{Second Interim Report} \\
        \large Group 4 - Resource Usage \\
        \vspace{1em}
        \large Will Hewes - wh365 \\ 
        Pembroke College \\ 
        \vspace{0.5em}
    \end{center}
}

\begin{document}
\includepdf[pages=-]{Handouts/IIA Project Coversheet Feedback Interim Report 2.pdf}
\maketitle
\hrule
\tableofcontents
\newpage

\section{Introduction}
\label{sec:Introduction}
% This section should give a detailed introduction to the changes
% you have implemented for your role on the team 
% (power/energy, time, or resource efficiency).

This project involves the collaborative design and implementation 
of a RISC-V processor on an iCE40 FPGA device. 
The overarching aim is to optimise 
the processor design with regard to its 
performance, power dissipation, and resource usage, 
while balancing trade-offs between these three aspects. 
Each team member is responsible for focusing on one of the three aspects.
My reports will focus on the resource usage of the design.

This report outlines the initial work carried out to reduce resource usage 
in the RISC-V processor implementation. 
The goal has been to reduce the number of LUTs and flip-flops consumed
by the processor while minimising 
excessive power dissipation or performance degradation.
These modifications form part of a broader group effort targeting performance, 
power consumption, and resource usage. 
Coordination has been ongoing to ensure our changes are compatible and 
contribute towards a coherent final design.

The analysis has concerned the
\texttt{bubblesort} program due to its increased complexity
relative to \texttt{hardwareblink} and \texttt{softwareblink},
allowing the processor to be pushed further and 
offering more representative resource utilisation patterns.
Preliminary synthesis results indicated that the original design 
made no use of available DSP blocks, 
instead relying heavily on general-purpose LUTs for arithmetic operations.

\section{Preliminary Results}
\label{sec:Preliminary_Results}
% This section should provide detailed quantitative information and results so far
All tables provided will be in order of the changes that were made
and displayed in the Appendix.
The total reduction in resource usage is 
shown in Table \ref{tab:modification_reductions}.
\subsection{Baseline}
\label{sec:Baseline}

Having performed the baseline analysis in the previous interim report,
the resource usage for the provided processor on the \texttt{bubblesort}
algorithm is shown in Table \ref{tab:baseline} the Appendix.

In the baseline design, 
synthesis of the unmodified processor design showed that approximately 
58\% of the available logic cells on the iCE40 UP5K were in use,
along with two thirds of the available Block RAMs. 
This figure indicates limited headroom for 
further feature integration or pipeline complexity, 
motivating efforts to streamline key datapaths and control logic. 
In particular, excessive use of general-purpose logic for ALU operations and CSR registers, 
along with unnecessarily wide control signal buses, 
were identified as contributors to high logic utilisation. 
Forwarding logic and arithmetic duplication remain areas for future optimisation.
Additionally, the design made no use of the available DSPs,
instead implementing its adder functions through LUTs.
These observations guided the subsequent design changes detailed in this report.

\subsection{Program Counter Clock Gating}
\label{sec:Program_Counter_Clock_Gating}

To reduce unnecessary switching activity in the control path, 
support for gated updates to the program counter was introduced. 
This involved modifying the \texttt{program\_counter.v} module 
to accept a \texttt{write\_enable} signal, 
allowing updates to be conditionally applied 
only when pipeline progression is required. 
The cpu.v module was updated accordingly to assert \texttt{write\_enable} 
whenever the stall signal is low, 
ensuring that the PC is held stable during pipeline stalls.

This change helps avoid logic toggling during idle or stalled cycles. 
This primarily has an impact on the power consumption, 
allowing the board to reduce unnecessary dynamic power dissipation 
caused by repeated switching in the program counter logic during pipeline stalls. 
The resource utilisation can be seen in 
Table \ref{tab:Program_Counter} in the Appendix,
helping free up 9 logic cells.
Although the impact on LUT usage is minimal, 
this is part of a broader strategy to reduce switching 
by selectively enabling pipeline register updates.

\subsection{Removal of CSR Logic}
\label{sec:CSR}

One of the most significant changes made was the complete removal of the 
Control and Status Register (CSR) logic. 
Control and Status Registers are optional RISC-V components used to 
support operating system-level features such as 
interrupt handling, timer access, and machine state tracking.

In the baseline design, CSR handling added extra complexity to the decode stage, 
required new control signals for accessing CSRs, 
and introduced a dedicated register file to store CSR values. 
This register file was implemented using block RAM, 
which increased resource usage.
As CSR instructions are not required 
for correct operation at this level of implementation, 
this functionality could be removed for the purposes of resource optimisation.
Since the benchmark programs execute purely in machine mode without 
requiring interrupts, exceptions, or privileged features, 
no CSR-related mechanisms are used, 
and their removal does not affect functional correctness.

The removal involved disabling CSR detection 
in \texttt{control\_unit.v} 
and replacing related wires with fixed constants.
Registers previously driven by CSR outputs were redirected to bypass logic 
or default values, ensuring that the pipeline remained functionally correct
without CSR support.

This change resulted in a notable reduction in logic complexity and 
created a marked drop in block RAM usage. 
As seen in Table \ref{tab:CSR} in the Appendix,
the logic cell usage dropped from 3064 to 2905, 
and the RAM usage dropped from 20 to 12,
freeing up an additional 26\% of the available block RAM.
It also simplifies the decode and 
execution stages, reducing combinational logic depth. 
However, it does remove support for future system-level extensions 
that rely on CSR functionality, and would need to be revisited if 
such features are later required.

\subsection{Trimming Control Signal Width}
\label{sec:Signal_Width}

While reviewing how the processor decodes instructions and carries them out, 
a closer examination of the logic and signals involved in these stages was performed
to find areas where resources could be saved.
It was identified that the control signal wire \texttt{cont\_mux\_out} 
was defined as a 32-bit bus, despite only 11 bits being used downstream. 
To simplify the design and reduce unnecessary routing, 
the width was reduced to match actual usage, 
and the custom \texttt{mux2to1\_11bit} module was 
added to represent this.

This change eliminates unused signal bits, 
improving signal clarity and slightly reducing synthesis complexity. 
As seen in Table \ref{tab:Signal_Width} in the Appendix,
the logic cell usage dropped from 2905 to 2891.
This contributes to improved maintainability 
and more efficient use of the FPGA's routing resources. 
Removing unused logic also helps the synthesiser 
more aggressively optimise signal propagation and packing.
This change is relatively minor in terms of immediate resource savings, 
but it illustrates a useful strategy for streamlining the design. 
By tightening signal widths and removing unused control bits, 
routing congestion can be reduced and synthesis complexity lowered. 
While the impact is small, applying this kind of low-level refinement 
across the processor can contribute meaningfully to overall optimisation. 
However, identifying such inefficiencies is time-consuming, 
as it often requires manual tracing of 
signals across modules and pipeline stages.

\section{Potential Risks}
\label{sec:Potential_Risks}
% This section should outline any challenges you ran into, 
% potential risks you see going into the final week, 
% and any steps you are taking to mitigate those risks.

While the changes implemented so far have improved logic utilisation 
without degrading performance significantly, several risks remain that could impact 
the stability or scalability of the processor design.

Firstly, future attempts to offload arithmetic operations to DSP blocks 
may reduce LUT usage, but will require careful mapping 
to avoid bottlenecks or timing degradation. 
Any overuse of these blocks could limit parallelism or 
necessitate costly logic workarounds. 
As of this stage, no DSP blocks are in use.

The most significant change to the implementation was the removal of CSR logic.
Though this has a very positive impact on the LUT usage and in particular
on the reduction of block RAMs required in the design,
it has implications with regard to 
future extensibility and system-level compatibility. 
By removing CSR support, the processor is no longer capable of 
handling interrupts, errors, or access control, 
which limits its applicability in more complex applications 
such as systems that use timers, interrupts, 
or need to manage multiple tasks

Lastly, clock gating introduced in the pipeline registers assumes 
correct stalling logic. 
If the stall signal is incorrectly asserted or delayed, it could lead to 
data hazards, incorrect register values, or pipeline deadlock. 
This risk increases as the pipeline control logic becomes more intricate.

Continued validation through simulation and synthesis will be necessary 
to ensure that further optimisation efforts do not compromise correctness 
or long-term maintainability.

\section{Future Work}
\label{sec:Future_Work}

Several areas remain that could be explored to further reduce resource usage:

\begin{itemize}[noitemsep]
    \item Simplifying large logic blocks using \texttt{generate} 
    statements in Verilog to reduce duplicated code.
    \item Reducing repeated comparator logic in the forwarding unit 
    by reusing shared comparison circuits.
    \item Using the FPGA's DSP blocks for arithmetic operations 
    to free up logic cells and improve efficiency.
    \item Cleaning up and simplifying control signal 
    wiring around the register file.
\end{itemize}

Each of these changes will be assessed for impact on logic usage
and the other performance metrics.


\section{Conclusion}
\label{sec:Conclusion}

This report outlines a series of logic and memory optimisations applied 
to the baseline RISC-V processor design. 
Through targeted changes - clock gating, CSR removal, 
and control signal width trimming - 
a cumulative reduction of 182 logic cells and 8 block RAMs was achieved 
without degrading functionality. 
These gains improve the design's flexibility 
for future performance or power-directed improvements.

The most immediate modification planned is to shift 
the adder operation to DSP blocks, which is expected to free up LUTs, 
reduce routing congestion, and improve overall timing performance.

\newpage
\appendix
%Use this section to include diagrams, Verilog or C code, etc
\section{Resource Usage Data}

\begin{table}[H] 
    \centering
    \begin{tabularx}{0.6\textwidth}{X c c}
        \toprule
        Modification & LUTs & Block RAMs \\ \midrule
        Baseline $\rightarrow$ PC Gating & $9$ & $0$ \\
        PC Gating $\rightarrow$ CSR Removal & $159$ & $8$ \\
        CSR Removal $\rightarrow$ Signal Width Fix & $14$ & $0$ \\ \midrule
        \textbf{Total Reduction} & \textbf{$182$} & \textbf{$8$} \\
        \bottomrule
    \end{tabularx}
    \caption{Individual resource reductions following each modification}
    \label{tab:modification_reductions}
\end{table}


\begin{table}[H] 
    \centering
    \begin{tabularx}{0.6\textwidth}{X c c}
        \toprule
        Resource type & Used & \% of total \\ \midrule
        Logic cells (\texttt{ICESTORM\_LC}) & 3073 / 5280 & 58\,\% \\
        Block RAMs (\texttt{ICESTORM\_RAM}) & 20 / 30 & 66\,\% \\
        IO buffers (\texttt{SB\_IO}) & 8 / 96 & 8\,\% \\
        Global buffers (\texttt{SB\_GB}) & 5 / 8 & 62\,\% \\
        HF oscillators (\texttt{ICESTORM\_HFOSC}) & 1 / 1 & 100\,\% \\
        \bottomrule
    \end{tabularx}
    \caption{Baseline report}
    \label{tab:baseline}
\end{table}

\begin{table}[H] 
    \centering
    \begin{tabularx}{0.6\textwidth}{X c c}
        \toprule
        Resource type & Used & \% of total \\ \midrule
        Logic cells (\texttt{ICESTORM\_LC}) & 3064 / 5280 & 57\,\% \\
        Block RAMs (\texttt{ICESTORM\_RAM}) & 20 / 30 & 66\,\% \\
        IO buffers (\texttt{SB\_IO}) & 8 / 96 & 8\,\% \\
        Global buffers (\texttt{SB\_GB}) & 5 / 8 & 62\,\% \\
        HF oscillators (\texttt{ICESTORM\_HFOSC}) & 1 / 1 & 100\,\% \\
        \bottomrule
    \end{tabularx}
    \caption{Report after adding clock gating to the Program Counter module}
    \label{tab:Program_Counter}
\end{table}

\begin{table}[H] 
    \centering
    \begin{tabularx}{0.6\textwidth}{X c c}
        \toprule
        Resource type & Used & \% of total \\ \midrule
        Logic cells (\texttt{ICESTORM\_LC}) & 2905 / 5280 & 57\,\% \\
        Block RAMs (\texttt{ICESTORM\_RAM}) & 12 / 30 & 40\,\% \\
        IO buffers (\texttt{SB\_IO}) & 8 / 96 & 8\,\% \\
        Global buffers (\texttt{SB\_GB}) & 5 / 8 & 62\,\% \\
        HF oscillators (\texttt{ICESTORM\_HFOSC}) & 1 / 1 & 100\,\% \\
        \bottomrule
    \end{tabularx}
    \caption{Report after removing Control and Status Register (CSR) logic}
    \label{tab:CSR}
\end{table}

\begin{table}[H] 
    \centering
    \begin{tabularx}{0.6\textwidth}{X c c}
        \toprule
        Resource type & Used & \% of total \\ \midrule
        Logic cells (\texttt{ICESTORM\_LC}) & 2891 / 5280 & 57\,\% \\
        Block RAMs (\texttt{ICESTORM\_RAM}) & 12 / 30 & 40\,\% \\
        IO buffers (\texttt{SB\_IO}) & 8 / 96 & 8\,\% \\
        Global buffers (\texttt{SB\_GB}) & 5 / 8 & 62\,\% \\
        HF oscillators (\texttt{ICESTORM\_HFOSC}) & 1 / 1 & 100\,\% \\
        \bottomrule
    \end{tabularx}
    \caption{Report after removing unused control signal width}
    \label{tab:Signal_Width}
\end{table}

\includepdf[pages=3-]{Reports/wh365-interim-report-1.pdf}

\end{document}