\documentclass[a4paper,10pt]{article}
% Use ctrl + alt + V to view live pdf

% Packages
\usepackage[utf8]{inputenc} % For encoding
\usepackage[T1]{fontenc} % Better handling of accented characters and hyphenation
\usepackage{microtype} % Improves spacing and justification
\usepackage{amsmath, amssymb} % For equations and symbols
\usepackage{graphicx} % For including graphics/images
\usepackage{caption} % For customizing figure and table captions
\usepackage{subcaption} % For subfigures and subcaptions
\usepackage{float} % For fixing figure and table positions
\usepackage{booktabs} % For professional-looking tables
\usepackage{siunitx} % For consistent typesetting of units and numbers
\usepackage[margin=2cm]{geometry} % Adjusts page margins
\usepackage{fancyhdr} % For custom headers and footers
\usepackage{lmodern} % For a professional-looking font (main body font)
\usepackage{titlesec} % For title customization
\usepackage{array} % For custom table formatting
\usepackage[colorlinks=true, linkcolor=black, urlcolor=black]{hyperref} % Colored links without boxes
\usepackage{cleveref} % For improved cross-referencing    
\usepackage{multirow}
\usepackage{enumitem}
\usepackage{listings}
\usepackage{xcolor}
\usepackage{textcomp}
\usepackage{tabularx}
\usepackage{changepage}
\usepackage{tikz}
\usepackage{pdfpages}
\usetikzlibrary{shapes.geometric, arrows}
\newcolumntype{Y}{>{\centering\arraybackslash}X}
% Reduce spacing before and after \section
% \titlespacing{\section}{0pt}{1.0em}{0.5em}
% Reduce spacing before and after \subsection
% \titlespacing{\subsection}{0pt}{0.8em}{0.3em}


\lstdefinestyle{vhdl-style}{
    language=VHDL,
    basicstyle=\ttfamily\footnotesize,
    keywordstyle=\bfseries\color{blue},
    commentstyle=\itshape\color{gray},
    stringstyle=\color{red},
    numbers=left,
    numberstyle=\tiny\color{gray},
    stepnumber=1,
    breaklines=true,
    showstringspaces=false,
    frame=single
}
\lstset{style=vhdl-style}
\lstset{captionpos=b}
\lstset{basicstyle=\ttfamily\scriptsize} 
\renewcommand{\lstlistingname}{Program}

% Custom settings
\pagestyle{fancy}
\fancyhf{}
\fancyhead[L]{\textit{GB3 - Risc-V Processor}} % Header left
\fancyhead[R]{\textit{Will Hewes - wh365}} % Header right 
\fancyfoot[C]{\thepage} % Footer center
\setlength{\headheight}{15pt} % Header height
\setlength{\parindent}{0em} % Indentation for paragraphs
\setlength{\parskip}{0.2em} % Add spacing between paragraphs
\setlength{\abovedisplayskip}{0.5em}
\setlength{\belowdisplayskip}{0.5em}
\setlength{\abovedisplayshortskip}{0.5em}
\setlength{\belowdisplayshortskip}{0.5em}
\setlist{topsep=0em, partopsep=0em, itemsep=0em, parsep=0em}

\graphicspath{{Images/}}

% \renewcommand{\arraystretch}{1.2}

% Title formatting
\renewcommand{\maketitle}{
    \begin{center}
        \LARGE \textbf{ENGINEERING TRIPOS PART IIA} \\ 
        \vspace{0.5em}
        \Large \textbf{GB3 - Risc-V Processor} \\ 
        \vspace{0.5em}
        \textbf{Final Report} \\
        \large Group 4 - Resource Usage \\
        \vspace{1em}
        \large Will Hewes - wh365 \\ 
        Pembroke College \\ 
        \vspace{0.5em}
    \end{center}
}

\begin{document}
\pagenumbering{gobble}
\includepdf[pages=-]{Handouts/IIA Project Coversheet Feedback Final Report.pdf}
\maketitle
\hrule
\tableofcontents
\newpage
\pagenumbering{arabic} \setcounter{page}{1}
% No more than 8 A4 sides, excluding appendices
% Include tables and descriptions in resource usage?

\section{Introduction}
\label{sec:Introduction}
% This section should give a detailed introduction to the changes
% you have implemented for your role on the team 
% (1 A4 side)

This project involves the collaborative design and implementation 
of a RISC-V processor on an iCE40 FPGA device. 
The overarching aim is to optimise 
the processor design with respect to its 
performance, power dissipation, and resource usage, 
while balancing trade-offs between these three aspects. 
Each team member is responsible for focusing on one of the three aspects
and my reports focus on the resource usage of the design.
This report is the final report for this project.

In the first report we evaluated three programs as a baseline against
which our improvements could be compared.
The first of these three porgrams was \texttt{hardwareblink},
a program which relied purely on combinational and sequential logic
to blink an LED on the FPGA,
bypassing the processor core entirely. 
As a result it had very low resource utilisation across the board.
The second program evaluated was \texttt{softwareblink},
which performed the same task but instead implemented through software,
utilising the sail-core processor.
As a result, the design used a much larger number of logic cells,
and also introduced the usage of a few other primitive cells such as 
block RAMs and DSP blocks.
The final program was \texttt{bubblesort}, which would perform a 
bubblesort algorithm and blink the LED to confirm it was finished.
Due to the increased complexity of the program,
the logic cell utilisation was greater than for \texttt{softwareblink},
but crucially the other primitives did not increase, 
as they relied primarily on their design within the verilog files.
The final test would be to process 
a new, unseen program under our implementation and evaluate it for its 
resource usage, performance and power dissipation.
As such we chose to base our testing primarily 
on the more complex \texttt{bubblesort},
allowing us to push the processor further,
offering a more representative view of our design.
It should be noted the other programs were still tested
to ensure functionality and check progress,
they were just less emblematic of our implementaions capabilities.

The baseline implementation on the FPGA showed several inefficiencies 
in its core design and made poor use of 
some of the resources available on the board. 
Before describing the changes made, it's helpful to briefly explain 
what those resources are and how they affect the processor.

Logic cells, or LCs, are the basic building blocks used to 
implement logic on the FPGA. 
They include components like look-up tables and flip-flops, 
and are used for almost everything, including 
arithmetic, control signals, and registers. 
Because almost everything in the design depends on logic cells in some way, 
reducing how many are used was the main goal throughout the project.
The changes that directly improved LC usage primarily came from
cleaning up registers, wires, and other redundant logic where possible.
One noteable example of this was reducing a multiplexing wire signal
from 32 bits down to 11 bits. These examples will be discussed in more detail
in the body of the report.

Other primitives available on the board are similarly powerful,
though they have a key difference - they are fixed for a given implementation
on the microprocessor.
This means that whereas LCs will increase and potentially become limiting
for a particularly complex program, 
other primitives such as block RAMs, DSP registers, and buffers
will not increase outside of the capacity of the board if designed properly.

Block RAMs are small dedicated memory blocks built into the FPGA. 
They're useful when a design needs to store more data 
than would be practical with flip-flops alone. 
In the baseline implementation, 
they were used by certain parts of the processor 
such as the register file and CSR logic. 
By removing CSR logic from the design,
the number of block RAMs used was reduced 
from 20 to 12 of the 30 available,
meaning that more memory functions could be moved to the RAM 
in order to preserve logic cells.
Unfortunately these changes could not be implemented due to time constraints,
but the plan was to ...

DSP blocks are specialised units designed for fast arithmetic. 
Instead of building adders and multipliers out of regular logic, 
the FPGA provides a few dedicated DSP blocks that are faster and use less space. 
The baseline design didn't use any of these, 
relying entirely on logic cells even for basic addition. 
One of the most significant alterations to the base implementation
was forcing the \textit{adder.v} to make use of these DSP blocks
for addition and subtraction, 
freeing up LUTs to be used in more sophisticated logic.
Implementing this change reduced the LC usage by ..., 
at the expense of 3 of the 8 DSP blocks available,
leaving room for further improvement in this regard.

In addition to these changes that were successfully implemented,
there are many that ...

\section{Design Strategy}
\label{sec:Design_Strategy}
% Describing the specific design strategy employed for your role in the team (0.5-1 A4 side)
% Timeline diagram?
% Include subsection on the baseline processor, both description of its structure and resource usage



\section{Design Description}
\label{sec:Design_Description}
% For the components you implemented for your team (regardless of your role) (2-3 A4 sides)
% Include diagram on DSP blocks?
% Mention how your changes affect the other aspects
% Summarise original behaviour, describe original behaviour, justify it

\section{Problems Encountered}
\label{sec:Problems_Encountered}
% Problems encountered by you and their technical solutions (1-2 A4 sides)
% Maybe mention minimising the impact of the others contributions i.e. pipelining
% Warnings, missing reports, LED not blinking after some tests
% Inability to test LED always due to Cheng having it, less flexibility?

\section{Test Procedure}
\label{sec:Test_Procedure}
% Test procedure for your components in the team's efforts (1-2 A4 sides)
% Testing procedure automated flow
% Test on bubblesort and software
% Manual testing?
% Git version control helping to keep track of affects on resource usage etc., ease of use

\section{Conclusion}
\label{sec:Conclusion}
% Conclusions and recommendations for further improvements in your design and evaluation
% Use this section to provide a retrospective on how your team coordinated amongst yourselves
% how that worked, and what you would do differently in the future (1-2 A4 sides)

\newpage
\appendix
%Use this section to include diagrams, Verilog or C code, etc
\section{Resource Usage Data}

\begin{table}[H] 
    \centering
    \begin{tabularx}{0.6\textwidth}{X c c}
        \toprule
        Modification & LUTs & Block RAMs \\ \midrule
        Baseline $\rightarrow$ PC Gating & $9$ & $0$ \\
        PC Gating $\rightarrow$ CSR Removal & $159$ & $8$ \\
        CSR Removal $\rightarrow$ Signal Width Fix & $14$ & $0$ \\ \midrule
        \textbf{Total Reduction} & \textbf{$182$} & \textbf{$8$} \\
        \bottomrule
    \end{tabularx}
    \caption{Individual resource reductions following each modification}
    \label{tab:modification_reductions}
\end{table}


\begin{table}[H] 
    \centering
    \begin{tabularx}{0.65\textwidth}{X c c}
        \toprule
        Resource type & Used & \% of total \\ \midrule
        Logic cells (\texttt{ICESTORM\_LC}) & 3073 / 5280 & 58\,\% \\
        Block RAMs (\texttt{ICESTORM\_RAM}) & 20 / 30 & 66\,\% \\
        IO buffers (\texttt{SB\_IO}) & 8 / 96 & 8\,\% \\
        Global buffers (\texttt{SB\_GB}) & 5 / 8 & 62\,\% \\
        HF oscillators (\texttt{ICESTORM\_HFOSC}) & 1 / 1 & 100\,\% \\
        \bottomrule
    \end{tabularx}
    \caption{Baseline report}
    \label{tab:baseline}
\end{table}

\begin{table}[H] 
    \centering
    \begin{tabularx}{0.65\textwidth}{X c c}
        \toprule
        Resource type & Used & \% of total \\ \midrule
        Logic cells (\texttt{ICESTORM\_LC}) & 3064 / 5280 & 57\,\% \\
        Block RAMs (\texttt{ICESTORM\_RAM}) & 20 / 30 & 66\,\% \\
        IO buffers (\texttt{SB\_IO}) & 8 / 96 & 8\,\% \\
        Global buffers (\texttt{SB\_GB}) & 5 / 8 & 62\,\% \\
        HF oscillators (\texttt{ICESTORM\_HFOSC}) & 1 / 1 & 100\,\% \\
        \bottomrule
    \end{tabularx}
    \caption{Report after adding clock gating to the Program Counter module}
    \label{tab:Program_Counter}
\end{table}

\begin{table}[H] 
    \centering
    \begin{tabularx}{0.65\textwidth}{X c c}
        \toprule
        Resource type & Used & \% of total \\ \midrule
        Logic cells (\texttt{ICESTORM\_LC}) & 2905 / 5280 & 57\,\% \\
        Block RAMs (\texttt{ICESTORM\_RAM}) & 12 / 30 & 40\,\% \\
        IO buffers (\texttt{SB\_IO}) & 8 / 96 & 8\,\% \\
        Global buffers (\texttt{SB\_GB}) & 5 / 8 & 62\,\% \\
        HF oscillators (\texttt{ICESTORM\_HFOSC}) & 1 / 1 & 100\,\% \\
        \bottomrule
    \end{tabularx}
    \caption{Report after removing Control and Status Register (CSR) logic}
    \label{tab:CSR}
\end{table}

\begin{table}[H] 
    \centering
    \begin{tabularx}{0.65\textwidth}{X c c}
        \toprule
        Resource type & Used & \% of total \\ \midrule
        Logic cells (\texttt{ICESTORM\_LC}) & 2891 / 5280 & 57\,\% \\
        Block RAMs (\texttt{ICESTORM\_RAM}) & 12 / 30 & 40\,\% \\
        IO buffers (\texttt{SB\_IO}) & 8 / 96 & 8\,\% \\
        Global buffers (\texttt{SB\_GB}) & 5 / 8 & 62\,\% \\
        HF oscillators (\texttt{ICESTORM\_HFOSC}) & 1 / 1 & 100\,\% \\
        \bottomrule
    \end{tabularx}
    \caption{Report after removing unused control signal width}
    \label{tab:Signal_Width}
\end{table}

\end{document}